\documentclass[10pt,conference]{IEEEtran}

% Including necessary packages
\usepackage{amsmath}
\usepackage{graphicx}
\usepackage{array}
\usepackage{enumitem}
\usepackage{float}
\usepackage{longtable}
\usepackage{geometry}
\geometry{margin=1in}

% Setting up the document
\begin{document}

% Starting document layout

% Defining table for static electric fields
\section*{Campos Eléctricos Estáticos}
\begin{table}[H]
    \centering
    \renewcommand{\arraystretch}{1.2}
    \resizebox{0.48\textwidth}{!}{
        \begin{tabular}{|c|c|}
            \hline
            Ley de Coulomb                  & $F = k_e \dfrac{|q_1 q_2|}{r^2}$                          \\
            \hline
            Constante de Coulomb            & $k_e = \dfrac{1}{4 \pi \epsilon_0}$                       \\
            \hline
            Campo eléctrico (punto)         & $\vec{E} = k_e \dfrac{q}{r^2} \hat{e}_r$                  \\
            \hline
            Campo eléctrico (línea)         & $\vec{E} = \dfrac{\lambda}{2 \pi \epsilon_0 r} \hat{e}_r$ \\
            \hline
            Campo eléctrico (plano)         & $\vec{E} = \dfrac{\sigma}{2 \epsilon_0} \hat{e}_n$        \\
            \hline
            Potencial eléctrico (general)   & $V = -\int \vec{E} \cdot d\vec{l}'$                       \\
            \hline
            Potencial eléctrico (punto)     & $V = k_e \dfrac{q}{r}$                                    \\
            \hline
            Trabajo eléctrico               & $W = q \Delta V$                                          \\
            \hline
            Energía potencial               & $U = \dfrac{q_1 q_2}{4 \pi \epsilon_0 r}$                 \\
            \hline
            Capacitancia (general)          & $C = \dfrac{Q}{V}$                                        \\
            \hline
            Capacitancia (placas paralelas) & $C = \epsilon \dfrac{A}{d}$                               \\
            \hline
            Capacitancia (coaxial)          & $C = \dfrac{2 \pi \epsilon l}{\ln(b/a)}$                  \\
            \hline
            Capacitancia (esférica)         & $C = 4 \pi \epsilon a$                                    \\
            \hline
        \end{tabular}
    }
    \caption{Fórmulas de campos eléctricos estáticos}
\end{table}

% Defining table for Gauss's laws
\section*{Leyes de Gauss}
\begin{table}[H]
    \centering
    \renewcommand{\arraystretch}{1.2}
    \resizebox{0.48\textwidth}{!}{
        \begin{tabular}{|c|c|}
            \hline
            \multicolumn{2}{|c|}{\textbf{Leyes de Gauss}}                                                   \\
            \hline
            Diferencial (eléctrico)  & $\nabla \cdot \vec{D} = \rho_v$                                      \\
            \hline
            Integral (eléctrico)     & $\oint \vec{D} \cdot d\vec{S}' = Q_{\text{enc}}$                     \\
            \hline
            Diferencial (magnético)  & $\nabla \cdot \vec{B} = 0$                                           \\
            \hline
            Integral (magnético)     & $\oint \vec{B} \cdot d\vec{S}' = 0$                                  \\
            \hline
            Ley de Gauss (eléctrico) & $\oint \vec{E} \cdot d\vec{S}' = \dfrac{Q_{\text{enc}}}{\epsilon_0}$ \\
            \hline
            Condición frontera (E)   & $\epsilon_1 E_{1n} - \epsilon_2 E_{2n} = \rho_s$                     \\
            \hline
            Condición frontera (D)   & $E_{1t} = E_{2t}$                                                    \\
            \hline
            Esfera (carga puntual)   & $\vec{E} = \dfrac{q}{4 \pi \epsilon_0 r^2} \hat{e}_r$                \\
            \hline
            Cilindro (línea carga)   & $\vec{E} = \dfrac{\lambda}{2 \pi \epsilon_0 r} \hat{e}_r$            \\
            \hline
        \end{tabular}
    }
    \caption{Fórmulas de las leyes de Gauss}
\end{table}

% Defining table for static magnetic fields
\section*{Campos Magnéticos Estáticos}
\begin{table}[H]
    \centering
    \renewcommand{\arraystretch}{1.2}
    \resizebox{0.48\textwidth}{!}{
        \begin{tabular}{|c|c|}
            \hline
            Ley Biot-Savart             & $d\vec{H} = \dfrac{I d\vec{l}' \times \hat{e}_r}{4 \pi r'^2}$ \\
            \hline
            Campo magnético (alambre)   & $\vec{H} = \dfrac{I}{2 \pi r} \hat{e}_\phi$                   \\
            \hline
            Fuerza magnética (alambre)  & $d\vec{F} = I d\vec{l}' \times \vec{B}$                       \\
            \hline
            Ley de Ampère (integral)    & $\oint \vec{H} \cdot d\vec{l}' = I_{\text{enc}}$              \\
            \hline
            Ley de Ampère (diferencial) & $\nabla \times \vec{H} = \vec{J}$                             \\
            \hline
            Condición frontera (B)      & $B_{1n} = B_{2n}$                                             \\
            \hline
            Condición frontera (H)      & $\vec{H}_{1t} - \vec{H}_{2t} = \vec{K}$                       \\
            \hline
            Fuerza de Lorentz           & $\vec{F} = q \vec{v} \times \vec{B}$                          \\
            \hline
            Campo magnético (solenoide) & $\vec{H} = \dfrac{N I}{l} \hat{e}_z$                          \\
            \hline
            Campo magnético (toroide)   & $\vec{H} = \dfrac{N I}{2 \pi r} \hat{e}_\phi$                 \\
            \hline
        \end{tabular}
    }
    \caption{Fórmulas de campos magnéticos estáticos}
\end{table}

% Defining table for electromagnetic waves - Part 1
\section*{Ondas Electromagnéticas - Parte 1}
\begin{table}[H]
    \centering
    \renewcommand{\arraystretch}{1.2}
    \resizebox{0.48\textwidth}{!}{
        \begin{tabular}{|c|c|}
            \hline
            Velocidad propagación     & $v = \dfrac{1}{\sqrt{\mu \epsilon}}$                                                               \\
            \hline
            Impedancia intrínseca     & $\eta = \sqrt{\dfrac{\mu}{\epsilon}}$                                                              \\
            \hline
            Ecuación onda (eléctrico) & $\dfrac{\partial^2 \vec{E}}{\partial z^2} = \mu \epsilon \dfrac{\partial^2 \vec{E}}{\partial t^2}$ \\
            \hline
            Relación E y H            & $\vec{E} = \eta \vec{H} \times \hat{e}_k$                                                          \\
            \hline
            Vector Poynting           & $\vec{P} = \vec{E} \times \vec{H}$                                                                 \\
            \hline
            Longitud de onda          & $\lambda = \dfrac{v}{f}$                                                                           \\
            \hline
        \end{tabular}
    }
    \caption{Fórmulas básicas de ondas electromagnéticas}
\end{table}

% Defining table for electromagnetic waves - Part 2
\section*{Ondas Electromagnéticas - Parte 2}
\begin{table}[H]
    \centering
    \renewcommand{\arraystretch}{1.2}
    \resizebox{0.48\textwidth}{!}{
        \begin{tabular}{|c|c|}
            \hline
            Atenuación (conductores) & $\alpha = \omega \sqrt{\dfrac{\mu \epsilon}{2} \left( \sqrt{1 + \left( \dfrac{\sigma}{\omega \epsilon} \right)^2} - 1 \right)}$ \\
            \hline
            Potencia promedio        & $P_{\text{avg}} = \dfrac{1}{2} \text{Re}(\vec{E} \times \vec{H}^*)$                                                             \\
            \hline
            Constante propagación    & $\gamma = \alpha + j\beta$                                                                                                      \\
            \hline
            Constante de fase        & $\beta = \omega \sqrt{\dfrac{\mu \epsilon}{2} \left( \sqrt{1 + \left( \dfrac{\sigma}{\omega \epsilon} \right)^2} + 1 \right)}$  \\
            \hline
            Reactancia inductiva     & $X_L = \omega L$                                                                                                                \\
            \hline
            Reactancia capacitiva    & $X_C = \dfrac{1}{\omega C}$                                                                                                     \\
            \hline
        \end{tabular}
    }
    \caption{Fórmulas avanzadas de ondas electromagnéticas}
\end{table}

% Defining table for Maxwell's equations
\section*{Leyes de Maxwell}
\begin{table}[H]
    \centering
    \renewcommand{\arraystretch}{1.2}
    \resizebox{0.8\textwidth}{!}{
        \begin{tabular}{|c|c|}
            \hline
            Ley de Gauss (eléctrico)     & $\nabla \cdot \vec{D} = \rho_v$                                                  \\
            \hline
            Ley de Gauss (magnético)     & $\nabla \cdot \vec{B} = 0$                                                       \\
            \hline
            Ley de Faraday (diferencial) & $\nabla \times \vec{E} = -\dfrac{\partial \vec{B}}{\partial t}$                  \\
            \hline
            Ley de Ampère-Maxwell (dif.) & $\nabla \times \vec{H} = \vec{J} + \dfrac{\partial \vec{D}}{\partial t}$         \\
            \hline
            Ley de Faraday (integral)    & $\oint \vec{E} \cdot d\vec{l}' = -\dfrac{d}{dt} \int \vec{B} \cdot d\vec{S}'$    \\
            \hline
            Ley de Ampère-Maxwell (int.) & $\oint \vec{H} \cdot d\vec{l}' = I + \dfrac{d}{dt} \int \vec{D} \cdot d\vec{S}'$ \\
            \hline
            Ley de Ampère (conductores)  & $\nabla \times \vec{H} = \vec{J} + \sigma \vec{E}$                               \\
            \hline
        \end{tabular}
    }
    \caption{Fórmulas de las leyes de Maxwell}
\end{table}

% Defining table for magnetostatics and transformers
\section*{Magnetostática y Transformadores}
\begin{table}[H]
    \centering
    \renewcommand{\arraystretch}{1.2}
    \resizebox{0.68\textwidth}{!}{
        \begin{tabular}{|c|c|}
            \hline
            Reluctancia                             & $\Re = \dfrac{l}{\mu A}$              \\
            \hline
            Flujo magnético                         & $\Phi = B A$                          \\
            \hline
            Densidad de flujo magnético             & $B = \mu H$                           \\
            \hline
            Ley de Faraday (fem)                    & $\mathcal{E} = -\dfrac{d\Phi}{dt}$    \\
            \hline
            Inductancia (solenoide)                 & $L = \dfrac{\mu N^2 A}{l}$            \\
            \hline
            Inductancia mutua                       & $M = k \sqrt{L_1 L_2}$                \\
            \hline
            Fuerza electromotriz inducida (general) & $\mathcal{E} = -N \dfrac{d\Phi}{dt}$  \\
            \hline
            Relación de transformación              & $\dfrac{V_2}{V_1} = \dfrac{N_2}{N_1}$ \\
            \hline
            Fem inducida (transformador)            & $V_2 = M \dfrac{dI_1}{dt}$            \\
            \hline
        \end{tabular}
    }
    \caption{Fórmulas de magnetostática y transformadores}
\end{table}

% Defining table for relations and constants
\section*{Relaciones y Constantes}
\begin{table}[H]
    \centering
    \renewcommand{\arraystretch}{1.2}
    \resizebox{0.48\textwidth}{!}{
        \begin{tabular}{|c|c|}
            \hline
            Permitividad del vacío  & $\epsilon_0 = 8.854 \times 10^{-12} \, \text{F/m}$                           \\
            \hline
            Permeabilidad del vacío & $\mu_0 = 4 \pi \times 10^{-7} \, \text{H/m}$                                 \\
            \hline
            Velocidad de la luz     & $c = \dfrac{1}{\sqrt{\mu_0 \epsilon_0}} \approx 3 \times 10^8 \, \text{m/s}$ \\
            \hline
            Relación D y E          & $\vec{D} = \epsilon \vec{E}$                                                 \\
            \hline
            Relación B y H          & $\vec{B} = \mu \vec{H}$                                                      \\
            \hline
            Permitividad compleja   & $\epsilon_c = \epsilon - j \dfrac{\sigma}{\omega}$                           \\
            \hline
        \end{tabular}
    }
    \caption{Relaciones y constantes fundamentales}
\end{table}

% Force a new page for the glossary
\clearpage

% Defining glossary as a list
\section*{Glosario}

\begin{itemize}[leftmargin=1.5cm]
    \item[$\alpha$] Coeficiente de atenuación
    \item[$\beta$] Constante de fase
    \item[$\gamma$] Constante de propagación
    \item[$\delta$] Profundidad de penetración
    \item[$\epsilon$] Permitividad del medio
    \item[$\epsilon_0$] Permitividad del vacío
    \item[$\epsilon_r$] Constante dieléctrica relativa
    \item[$\eta$] Impedancia intrínseca
    \item[$F$] Fuerza eléctrica
    \item[$\vec{E}$] Campo eléctrico
    \item[$\vec{D}$] Desplazamiento eléctrico
    \item[$\vec{H}$] Campo magnético
    \item[$\vec{B}$] Densidad de flujo magnético
    \item[$I$] Corriente eléctrica
    \item[$\vec{J}$] Densidad de corriente
    \item[$k_e$] Constante de Coulomb
    \item[$\lambda$] Densidad de carga lineal
    \item[$L$] Inductancia
    \item[$M$] Inductancia mutua
    \item[$\mu$] Permeabilidad del medio
    \item[$\mu_0$] Permeabilidad del vacío
    \item[$\mu_r$] Permeabilidad relativa
    \item[$\nu$] Frecuencia
    \item[$\Phi$] Flujo magnético
    \item[$\Re$] Reluctancia
    \item[$\rho_v$] Densidad de carga volumétrica
    \item[$\rho_s$] Densidad de carga superficial
    \item[$\vec{K}$] Corriente superficial
    \item[$q$] Carga eléctrica
    \item[$r$] Distancia radial
    \item[$\sigma$] Conductividad eléctrica
    \item[$v$] Velocidad de propagación
    \item[$V$] Potencial eléctrico
    \item[$\mathcal{E}$] Fuerza electromotriz
    \item[$\omega$] Frecuencia angular
    \item[$X_L$] Reactancia inductiva
    \item[$X_C$] Reactancia capacitiva
    \item[$A$] Área de sección transversal
    \item[$B_{\text{max}}$] Densidad de flujo máxima
    \item[$f$] Frecuencia
    \item[$k$] Coeficiente de acoplamiento
    \item[$k_e, k_h$] Constantes de pérdidas
    \item[$l$] Longitud
    \item[$N$] Número de espiras
    \item[$P$] Potencia
    \item[$t$] Espesor
    \item[$Z$] Impedancia
    \item[$\hat{e}_r, \hat{e}_\phi, \hat{e}_n, \hat{e}_k$] Vectores unitarios
\end{itemize}

\end{document}